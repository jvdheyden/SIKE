% !TeX root = ../thesis.tex
\section{Cryptosystem}
\subsection{The Supersingular Isogeny Diffie-Hellman protocol}

The idea of the SIKE cryptosystem is to establish a Diffie-Hellman key-exchange protocol based on isogenies between supersingular elliptic curves. 



\begin{enumerate}[(1)]
	\item We start by generating the \textit{public parameters}: Choose $e_A,e_B$ and look for a prime $p=2^{e_A}3^{e_B}f\pm1$. Set $q=p^2$ and find a supersingular curve $E$ over $\mathbb{F}_q$. Finally, select points $P_A,Q_A$ s.t. they generate the endomorphism $E[2^{e_A}]$ and $P_B,Q_B$ s.t. $\langle P_B,Q_B\rangle=E[3^{e_B}]$. %TODO: What prevents the points from having an order other than 3^{e_B} 2^{e_A}?
	\item Subsequently Alice and Bob choose \textit{secret parameters} $m_A,n_A$ and $m_B,n_B$ respectively to find the secret points $[m_A]P_A + [n_A]Q_A$ and $[m_B]P_B + [n_B]Q_B$. The groups generated by these points serve as the kernels for the secret isogenies $\phi_A : E \to E/A$ and $\phi_B : E \to E/B$.
	\item In order to allow Bob compute a secret point in E/A, Alice sends him $\phi_A(P_B)$ and $\phi_A(Q_B)$. Upon receival Bob then finds $T_B = \phi_A([m_B]P_B + [n_B]Q_B)$ in $E/A$ and uses $\langle T_B \rangle$ as kernel for $\phi^{\prime}_B : E/A \to E/BA$. Alice proceeds \textit{mutatis mutandis} with $T_A = \phi_B([m_A]P_A + [n_A]Q_A)$ and computes $\phi^{\prime}_A : E/B \to E/AB$ with $ker(\phi^{\prime}_A)= \langle T_A \rangle$.
	\item As shown in \ref{ch:correctness}, $E/AB \cong E/BA$ and therefore the shared secret key can be generated from $j(E/AB)=j(E/BA)$.
	
	It is easy to see that the protocol could be broken, if an attacker would be able to:
	
	\begin{itemize}[\textbullet]
		\item Use the knowledge of $E$ and $E/A$ to compute $\phi_A$.
		\item Use the knowledge of $\phi_A(P_B)$ and $P_B$ to compute $\phi_A$
		\item
	\end{itemize}
	
	%TODO: Schaubild anhängen
	
\end{enumerate} 


\subsection{Correctness}
\label{ch:correctness}


\subsection{Computational Details}
%TODO: Why is it easy to compute isogenies?
