\documentclass[handout]{beamer}
%TODO:
%performance metrics

% Choose how your presentation looks.
%
% For more themes, color themes and font themes, see:
% http://deic.uab.es/~iblanes/beamer_gallery/index_by_theme.html
%
\mode<presentation>
{
  \usetheme{Madrid}      % or try Darmstadt, Madrid, Warsaw, AISEC ...
  \usecolortheme{dolphin} % or try albatross, beaver, crane, ...
  \usefonttheme{default}  % or try serif, structurebold, ...
  \setbeamertemplate{navigation symbols}{}
  \setbeamertemplate{caption}[numbered]
} 

%\usepackage{fhgfont} funktioniert leider noch nicht
\usepackage[shortlabels]{enumitem} % counter styles for enumerations
\usepackage[font=footnotesize]{caption} %for attributing pictures
\usepackage[english]{babel}
\usepackage[utf8x]{inputenc}
\usepackage{braket} % dirac notation
\usepackage{amsmath} % math symbols
\usepackage{amssymb} % other symbols
\graphicspath{ {img/} }
\usepackage{svg} %insert svg
\usepackage{svg-extract} %insert svg
\usepackage{graphicx} % insert pdfs
\newenvironment{rcases} % for right braces
{\left.\begin{aligned}}
	{\end{aligned}\right\rbrace}

\title[SIKE]{SIKE - Supersingular Isogeny Key Encapsulation}
\author{Jonas von der Heyden}
\institute{FU Berlin}
\date{10.7.19}

\begin{document}
\newcommand{\source}[1]{\caption*{Source: {#1}} } %for attributing pictures
\begin{frame}
  \titlepage
\end{frame}

% Uncomment these lines for an automatically generated outline.
\begin{frame}{Outline}
  \tableofcontents
\end{frame}

\section{Introduction}

\begin{frame}{Introduction}

SIKE - Supersingular Isogeny Key Encapsulation\\
\vspace{5mm}
\textbf{The Good:}
\begin{enumerate}[(i)]
	\item Very small key sizes\pause
	\item No commutative group action\pause
	\item Potential for performance improvements\pause
\end{enumerate}
\vspace{5mm}
\textbf{The Bad:}
\begin{enumerate}[(i)]
	\item Comparably poor, yet practical performance\pause
	\item Very recent\pause
\end{enumerate}

\vfill
\textbf{The Ugly:} Concepts are so obscure and complicated that only mathematicians understand it.


\end{frame}
\begin{frame}{Comparison of DH, ECDH and SIDH}
\begin{figure} %TODO: DH an Tafel
	\centering
	\includegraphics[width=1\linewidth]{dh_ec_iso}
	\label{fig:dh_ec_iso}
\end{figure}
\end{frame}



\section{Mathematical primitive}

\subsection{Elliptic Curves}

\begin{frame}{Elliptic curves}
	Which of the figures is \textit{not} an elliptic curve?
	\begin{figure}
		\begin{minipage}{0.48\textwidth}
			\centering
			\includegraphics[width=.7\linewidth]{ellipse}
			\label{fig:ellipse}
		\end{minipage}\hfill
		\begin{minipage}{0.48\textwidth}
			\centering
			\includegraphics[width=.7\linewidth]{elliptic_curve}
			\label{fig:elliptic_curve}
		\end{minipage}
	\end{figure}
	\begin{figure}
	\begin{minipage}{0.48\textwidth}
		\centering
		\includegraphics[width=.7\linewidth]{elliptic_curve_c}
		\label{fig:elliptic_curve_c}
	\end{minipage}\hfill
	\begin{minipage}{0.48\textwidth}
		\centering
		\includegraphics[width=.7\linewidth]{elliptic_curve_fp}
		\label{fig:elliptic_curve_fp]}
	\end{minipage}
\end{figure}


	% image of ellipse
\end{frame}

\begin{frame}{Elliptic curves}
	Which of the figures is \textit{not} an elliptic curve?
\begin{figure}
	\begin{minipage}{0.48\textwidth}
		\centering
		\includegraphics[width=.7\linewidth]{ellipse}
		\caption{Ellipse}\label{fig:ellipse}
	\end{minipage}\hfill
	\begin{minipage}{0.48\textwidth}
		\centering
		\includegraphics[width=.7\linewidth]{elliptic_curve}
		\caption{Elliptic Curve over $\mathbb{R}$}\label{fig:elliptic_curve}
	\end{minipage}
\end{figure}
\begin{figure}
	\begin{minipage}{0.48\textwidth}
		\centering
		\includegraphics[width=.7\linewidth]{elliptic_curve_c}
		\caption{Elliptic Curve over $\mathbb{C}$}\label{fig:elliptic_curve_c}
	\end{minipage}\hfill
	\begin{minipage}{0.48\textwidth}
		\centering
		\includegraphics[width=.7\linewidth]{elliptic_curve_fp}
		\caption{Elliptic Curve over $\mathbb{F}_p$}\label{fig:elliptic_curve_fp]}
	\end{minipage}
\end{figure}

\end{frame}
\begin{frame}{Elliptic curves}
What is an elliptic curve?
\begin{itemize}[\textbullet]
	\item Historically, elliptic curves were used to calculate the circumference of an ellipse
	\item An elliptic curve is the set of solutions to a \textit{Weierstrass equation} of the form $Y^2=X^3+AX+B$
\end{itemize}

\begin{figure}
	\centering
	\includegraphics[width=.7\linewidth]{elliptic_curve}
	\label{fig:elliptic_curve}
\end{figure}
	
\end{frame}

\begin{frame}{Abelian Groups (G,+) over Elliptic Curves}
\begin{figure}
	\begin{minipage}{0.5\textwidth}
		\centering
		\includegraphics[width=1\linewidth]{P+Q}
		\caption{$P+Q=R^{\prime}$}\label{fig:p+q}
	\end{minipage}\hfill
	\begin{minipage}{0.48\textwidth}
		\centering
		\includegraphics[width=1\linewidth]{P+P}
		\caption{$P+P=R^{\prime}$}\label{fig:p+p}
	\end{minipage}
\end{figure}

\end{frame}
\begin{frame}{Abelian Groups (G,+) over Elliptic Curves}
\begin{figure}

		\centering
		\includegraphics[width=1\linewidth]{P-P}
		\caption{$P+P^{\prime}=P-P=\mathcal{O}$}\label{fig:p-p}


\end{figure}
\end{frame}
\begin{frame}{Abelian Groups (G,+) over Elliptic Curves}
	Let E be an elliptic curve. Then the addition law on E has the following properties:
	\begin{enumerate}[1.]
		\item $P + \mathcal{O} = \mathcal{O} + P$ (Identity)\pause
		\item $P + (-P) = \mathcal{O}$ (Inverse)\pause
		\item $(P + Q) + R = P + (Q + R) $ (Associative)\pause
		\item $P + Q = Q + P$ (Commutative)
	\end{enumerate}

\end{frame}

%\begin{frame}{Elliptic Curve Point Addition}
%
%\begin{itemize}[\textbullet]
%	\item Find the line intersecting E at P and Q\pause
%	\item If $P=Q$, find tangent of E with the same gradient like E at point P\pause
%	\item Plug line equation into E and solve to find the  point $R$ of intersection\pause
%	\item Flip y-coordinate to get $R^{\prime}$\pause
%\end{itemize}
%
%\end{frame}

%
%\begin{frame}{Elliptic Curve addition algorithm}
%	Let $E : Y^2 = X^3 + AX + B$ be an elliptic curve and let $P_1$ and $P_2$ be points on E.
%	\begin{enumerate}[1.]
%		\item If $P_1 = \mathcal{O}$, then $P_1 + P_2 = P_2$\pause
%		\item Otherwise, if $P_2=\mathcal{O}$, then $P_1 + P_2 = P_1$\pause
%		\item Otherwise, write $P_1 = (x_1,y_1)$ and $P_1 = (x_2,y_2)$\pause
%		\item If $x_1 = x_2$ and $y_1=-y_2$, then $P_1+P_2=\mathcal{O}$\pause
%		\item Otherwise, define \\
%		\qquad \qquad \qquad$\lambda =
%		\begin{cases}
%			\frac{y_2-y_1}{x_2-x_1}\text{\quad if }P_1 \neq P_2\\\pause
%			\\
%			\frac{3x_1^2 +A}{2y_1}\text{\quad if }P_1=P_2 
%		\end{cases}$\\\pause
%		\vspace{5mm}
%		and let $x_3=\lambda^2-x_1-x_2$  \hfill and $y_3=\lambda(x_1-x_3)-y_1$.\\
%		\vfill
%		Then $P_1+P_2=(x_3,y_3)$.
%	\end{enumerate}
%\end{frame}





%\begin{frame}{Why use supersingular curves}
%\begin{itemize}[\textbullet]
%	\item We can easily control their group structure which is required for efficient computation of isogenies
%	\item Their endomorphism ring is not commutative which  prevents some quantum attacks
%\end{itemize}
%\end{frame}

%\begin{frame}{ECDLP}
%\end{frame}



\subsection{Isogenies}
%\begin{frame}{Group homomorphisms}
%Properties of group homomorphisms:
%\begin{itemize}[\textbullet]
%	\item Given 2 groups (G,+), (H,$\oplus$) a group homomorphism is a function $\phi: G \to H$ such that $\phi(u + v) = \phi(u) \oplus \phi(v)$ \pause
%	\item $\phi(e_G) = e_H$ where $e_G$ and $e_H$ are the identity elements of G and H\pause
%	\item The kernel of $\phi$ is the set of elements in $G$ which are mapped to the identity in $H$:
%	$ker(\phi) \equiv \{u\in G:\phi(u)=e_H\}$
%\end{itemize}
%	 
%	
%	%TODO: Schaubild an Tafel
%	\begin{block}{Example}
%		$\phi: \mathbb{Z}\to\mathbb{Z}/3\mathbb{Z}$ is surjective and it's kernel consists of all elements in $\mathbb{Z}$ divisible by 3
%		
%		
%	\end{block}
%\end{frame}
%\begin{frame}{Types of Homorphisms}
%	\begin{itemize}[\textbullet]
%		\item Isomorphism: A group homomorphism that is bijective \pause
%		\item Endomorphism: A group homomorphism $\phi: G \to G$\pause
%		\item Isogeny: A group homomorphism $\phi : G \to H$ with a finite kernel
%	\end{itemize}
%
%\end{frame}
\begin{frame}{Endomorphism example}
\begin{itemize}[\textbullet]
	\item A point $P\in E$ satisfying $[m]P=\mathcal{O}$ is called a point of order m in the group E\pause
	\item Endomorphism $E[m] = \{P\in E: [m]P = \mathcal{O}\}$ is also called the m-torsion group of E and includes all points of order $m$ in $E$\pause
	\item $E[m] $ is a subgroup of E since if $P,Q\in E[m]$ then also $P+Q$ and $-P$
\end{itemize}	

\end{frame}


\begin{frame}{Isogenies of elliptic curves}  %TODO: Example 3 aus Galbraith an Tafel
Let $E_1,E_2$ be 2 elliptic curves over $\mathbb{F}_q$. Then an isogeny is a morphism $\phi: E_1 \to E_2$ s.t.:
\begin{itemize}[\textbullet]
	\item $\phi(0_{E_1})= 0_{E_2}$\pause
	\item $\#E_1(\mathbb{F}_q) = \#E_2(\mathbb{F}_q)$\pause
	\item there exists a dual isogeny $\hat{\phi}: E_2 \to E_1$\pause
	\item up to isomorphism it is uniquely defined by it's kernel $ker(\phi)$. Because a kernel is also a finite subgroup G of $E_1$ we can also say that $E_2$ is $E_1/G$\pause
	\item equivalently it is also uniquely defined up to isomorphism by it's j-invariant $j(E)=1728\frac{4A^3}{4A^3+27B^2}$
	%TODO: how does j-invariant work?
\end{itemize}
\end{frame}

\begin{frame}{Isogeny Example}
	Multiplication map $[n]: E \to E^{\prime}$ defined by $[n]P = P + P + ...+ P$ ($n$ times):
	\begin{itemize}[\textbullet]
		\item Maps 0 to itself
		\item $ker[n] = E[n]$, therefore kernel is finite 
		\item Isogeny [n] can be calculated with elliptic curve addition algorithm %Beispiel an Tafel?
		
	\end{itemize}

\end{frame}





\section{Cryptosystem}

\subsection{SIDH protocol}

\begin{frame}{The SIDH protocol: Main idea}
Idea: Create Diffie-Hellman exchange over isogenies of supersingular elliptic curves %DLDH an Tafel zeichnen
\begin{figure}
	\begin{minipage}{0.75\textwidth}
		\centering
		\includegraphics[width=1\linewidth]{sidh_costello_1}
		\caption{Supersingular Isogeny Diffie-Hellman}\label{fig:dldh}
	\end{minipage}\hfill
%	\begin{minipage}{0.48\textwidth}
%		\centering
%		\includegraphics[width=1\linewidth]{SIDH}
%		\caption{Supersingular Isogeny Diffie-Hellman}\label{fig:sidh}
%	\end{minipage}
\end{figure}

\end{frame}


\begin{frame}{The SIDH protocol: Setting up parameters}
\begin{enumerate}[1.]
	\item Generate public parameters:
	\begin{enumerate}[(a)]
		\item Choose $2^{e_A},3^{e_B}$, search for a prime $p=2^{e_A}3^{e_B}f- 1$\pause %and compute $q=p^2$
		\item Set $q=p^2$ and construct elliptic curve $E$ over $\mathbb{F}_q$ of cardinality $(2^{e_A}3^{e_B}f)^2$\pause %wie geht das?
		\item Find bases $P_A,Q_A$ and $P_B,Q_B$ s.t. $\langle P_A,Q_A\rangle = E[2^{e_A}]$ and $\langle P_B,Q_B\rangle = E[3^{e_B}]$ \pause  %TODO: wie schwierig ist das?
		\item Note that $\#E[2^{e_A}] =(2^{e_A})^2$
	\end{enumerate}
	\item Generate secret parameters:
	\begin{enumerate}[(a)]
		\item Alice picks 2 random secret elements $m_A,n_A < 2^{e_A}$, Bob picks 2 random secret elements $m_B,n_B < 3^{e_B}$\pause
		\item Alice computes secret isogeny $\phi_A : E \to E/A$ with kernel $\langle [m_A]P_A+[n_A]Q_A\rangle$, Bob computes secret isogeny $\phi_B : E \to E/B$ with kernel $\langle [m_B]P_B+[n_B]Q_B\rangle$
	\end{enumerate}

\end{enumerate}
\end{frame}

\begin{frame}{The SIDH protocol: Commutative Action}

\begin{enumerate}[1.]
	\item Compute $\phi_A^{\prime},\phi_B^{\prime}$

	\begin{enumerate}[(a)]
		\item Alice computes the image $\{\phi_A(P_B),\phi_A(Q_B)\} \subset E/A$ and sends it to Bob and vice versa\pause
		\item Now Alice is able to compute the point $T_A=[m_A]\phi_B(P_A) + [n_A]\phi_B(Q_A)$ in $E/B$\pause
		\item Next Alice computes the isogeny $\phi_A^{\prime}$ with kernel $\langle T_A \rangle$ to the curve $E/AB$\pause
		\item Bob proceeds \textit{mutatis mutandis}. On the next slide we will show that $\langle T_B \rangle  = ker(\phi_B^{\prime})=ker(\phi_A^{\prime})=\langle T_A \rangle$ and therefore $E/AB \simeq E/BA$.
	\end{enumerate}
	\item Alice and Bob compute the j-invariant $j(E/AB)=j(E/BA)$ to form a shared secret key 
\end{enumerate}
\end{frame}

%\begin{frame}{The SIDH protocol: Correctness}
%\textbf{Question:} Why is $ker(\phi_B^{\prime})=ker(\phi_A^{\prime})$?\pause\\
%\vspace{5mm}
%\textbf{Lemma 1:} Let $\phi_1: G_1 \to G_2$ and $\phi_2: G_2 \to G_3$ be group homomorphisms with $ker(\phi_1) = \langle K_1 \rangle$ and $ker(\phi_2) = \phi_1(\langle K_2 \rangle)$. Then for $\phi:G_1 \to G_3$ defined by $\phi=\phi_2 \circ \phi_1$ we have $ker(\phi) =\langle K_1,K_2\rangle$.\pause\\ %beweis an Tafel
%\vspace{5mm}
%
%\textbf{Theorem 2:} $ker(\phi_B^{\prime} \circ \phi_A) = ker(\phi_A^{\prime} \circ \phi_B)$\\
%\textbf{Proof:}
%	\begin{enumerate}[(i)]
%		\item $ker(\phi_B^{\prime})=\phi_A(\langle [m_B]P_B + [n_B]Q_B)\rangle)$\pause
%		\item $ker(\phi_A)=\langle [m_A]P_A + [n_A]Q_A\rangle$\pause
%		\item $ker(\phi_B^{\prime} \circ \phi_A) \stackrel{\text{Lemma 1}}{=} \langle [m_A]P_A + [n_A]Q_A, [m_B]P_B + [n_B]Q_B \rangle$\pause
%		\item The application of Lemma 1 on $\phi_A^{\prime} \circ \phi_B$ yields the same kernel
%	\end{enumerate}
%
%\vfill
%
%Since isogenies are uniquely defined by their kernel, $E/AB \simeq E/BA$.
%
%\end{frame}

%\begin{frame}{The SIDH protocol: Visualization} %besser an due Tafel malen!
%\begin{figure}
%	\centering
%	\includegraphics[width=0.7\linewidth]{SIDH_big}
%	\caption{SIDH}
%	\label{fig:sidh_big}
%\end{figure}
%\end{frame}



\subsection{Computation Details and Efficiency}

%\begin{frame}{Efficiently computing isogenies}
%
%
%	\textbf{Problem:} Computing $\phi_G : E \to E/A$ from $E$ and $A$ takes time polynomial in $\#A$. Since $2^{e_A} \in \mathcal{O}(\sqrt{p})$, this is not acceptable.\pause\\
%	\vfill
%	\textbf{Solution:} Compute isogeny of large degree as a composition of isogenies of small degree. Then $\phi = \phi_1 \circ \dots \circ \phi_k \circ [n]$ where $\phi_1 \circ \dots \circ \phi_k$ are isogenies of prime degree that are defined over $\mathbb{F}_q$ and $deg(\phi) = n^2 \prod^k_{i=1} deg(\phi_i)$.\\
%	 While computing the composition of $t$ isogenies of degree $t$ takes time proportional to $t$, the cost of computing the composition in a single step is in $\mathcal{O}(2^t)$\\
%	 
%	%TODO: Grund für supersingular curves: rational torsion points that makes efficient computation possible	
%
%\end{frame}
%\begin{frame}{The SIKE protocol}
%
%How does the SIKE specification for NIST PQ competition differ from SIDH?
%
%\begin{itemize}[\textbullet]
%	%	\item SIKE is a key encapsulation scheme that uses the SIDH protocol to generate a shared key between 2 parties % TODO: vielleicht key encapsulation kurz zeigen
%	\item SIKE is IND-CPA-secure and IND-CCA-secure %TODO: noch herausfinden warum
%	\item SIKE uses compression to reduce key sizes and computes private keys in constant time to defend against side-channel attacks
%\end{itemize}
%
%\end{frame}
\section{Areas of use}

\begin{frame}{Areas of use}
SIKE is characterized by having a very small key size but being fairly inefficient in the key-generation. Therefore it seems to be most suitable for:
\begin{itemize}[\textbullet]
	
	\item Securing applications where bandwidth is a more precious commodity than computational cycles, e.g. Bitcoin and Tor\pause
	\item Securing TLS once further research has resulted in better performance\pause
	\item Use in a hybrid cryptosystem combining untested post-quantum-cryptography with a conventional encryption scheme since it can use the same elliptic curve like ECDH	\pause
\end{itemize}
At this stage, it is not suitable for use in embedded devices or microcontrollers due to high energy demand.
\end{frame}

\begin{frame}{Comparison of key sizes and computational cycles}
\begin{figure} 
	\centering
	\includegraphics[width=1\linewidth]{performance}
	\label{fig:performance}
\end{figure}
\end{frame}

\section{Security \& Complexity}
%
%% SIKE spec 4.1
%% perfect forward secrecy
%% reuse of keys
%% sidh security paper p.5
%% side channel & error attacks?
%% galbraith: security of all schemes of this type depends on the difficulty of computing the endomorphism ring of a supersingular elliptic curve
%%why supersingular: because of endomorphism ring
\begin{frame}{Security assumptions} %Mündlich: Es wird angenommen, dass diese Probleme schwer sind.
\begin{enumerate}[1.]
	\item \textbf{Computational Supersingular Isogeny problem:}\\
	Let $\phi_A:E\to E/A$ be an isogeny with $ker(\phi_A)=\langle [m_A]P_A+[n_A]Q_A\rangle$. Given $E/A$ and the values $\phi_A(P_B),\phi_A(Q_B)$ find a generator $R_A$ of $\langle [m_A]P_A+[n_A]Q_A\rangle$\pause
	\vspace{5mm}
	\item \textbf{Computational Supersingular Diffie-Hellman problem:}\\ 
	Let $\phi_A:E\to E/A$ be an isogeny with $ker(\phi_A)=\langle [m_A]P_A+[n_A]Q_A\rangle$ and let $\phi_B:E\to E/A$ be an isogeny with $ker(\phi_B)=\langle [m_B]P_B+[n_B]Q_B\rangle$
	Given $E/A,E/B$ and the points $\phi_A(P_B),\phi_A(Q_B),\phi_B(P_A),\phi_B(Q_A)$ find the j-invariant of $E/\langle [m_A]P_A+[n_A]Q_A,[m_B]P_B+[n_B]Q_B \rangle$\pause
	\vfill
	It is being assumed that both problems are hard. There is no reduction to an NP-complete problem.
\end{enumerate}
\end{frame}

\begin{frame}{Known attacks: claw algorithm}
The best known attack against SIKE is the "claw algorithm":

\begin{enumerate}[(i)]
	\item Given $\phi : E \to E_A$ with $deg(\phi) = 2^{e_A}$, find $\phi$\pause
	\item Start with computing all $2^{e_A/_2}$-isogenies of one curve in $\mathcal{O}(p^{1/4})$ space and time\pause
	\item Then compute all $2^{e_A/_2}$-isogenies of the other curve and look for a collision\pause
	\item Overall space and time requirement is in $\mathcal{O}(p^{1/4})$ on a classical computer and $\mathcal{O}(p^{1/6})$ quantumly
\end{enumerate}
\end{frame}
%\begin{frame}{Known attacks: endomorphism ring}


%\begin{itemize}[\textbullet]
%	\item The endomorphism ring of E is the set of isogenies from E to itself: $End(E) = \{\phi:E\to E\} \cup \{0\}$\pause
%	\item Addition of isogenies is defined using elliptic curve addition as $(\phi_1+\phi_2)(P) =  \phi_1(P) + \phi_2(P)$\pause
%	\item Multiplication is the composition $\phi_1 \circ \phi_2$\pause
%	\item Given $E$ and $End(E)$, computing $\phi: E \to E'$ can be reduced to computing $End(E')$. This is shown to be possible in subexponential time on quantum computers, if E is ordinary\pause
%	\item Unlike ordinary curves, the endomorphism ring of supersingular curves is non-commutative and is therefore not susceptible to the quantum attack
%\end{itemize}
%
%
%
%\end{frame}

%\begin{frame}{Resistance against side-channel and fault attacks}
%Attack vectors for timing, power and fault analysis:
%\begin{itemize}[\textbullet]
%	\item Computation of the hidden kernel point
%	\item Computation of the secret isogeny
%\end{itemize}
%
%\vfill
%
%As a countermeasure all secret values are being computed using constant-time-functions. However, this does not defend against fault injection attacks.
%
%
%\end{frame}
%
%\begin{frame}{Passive and active security}
%
%SIKE is IND-CPA and IND-CCA secure.
%
%\begin{itemize}[\textbullet]
%	\item \textbf{IND-CPA}: SIDH can be modified to be secure against a passive attacker by converting it into an ElGamal scheme
%	\item \textbf{IND-CCA}: By using a Key-Encapsulation-Mechanism (KEM) it is possible to prevent any active attacks as well
%\end{itemize}
%
%%costello slides
%\end{frame}

\section{Discussion}

\begin{frame}{Conclusion}

SIKE summary
\begin{itemize}[\textbullet]
	
	\item All known quantum and classical attacks are exponential\pause
	\item Security research on isogeny-based cryptosystems might not be as extensive as for less complicated cryptosystems\pause
	\item Great candidate for hybrid cryptosystems\pause
	\item Current performance worse than alternatives, but potential for improvements\pause
\end{itemize}
\vfill
Open Questions
\begin{itemize}[\textbullet]
	\item How can we qualify the reliability of SIKE's security assumptions?\pause
	\item Why aren't more cryptosystems based on one of the more than 1000 NP-complete problems? 
\end{itemize}
%Nachteil: sehr kompliziert, daher ist Sicherheitsanalyse schwierig (z.B. für quantum researcher)Parameter müssen mit Bedacht ausgewählt werden (SIDH spec 1.3.2.)
% SIKE spec chapter 6
\end{frame}

\begin{frame}{Questions?}

\end{frame}

\end{document}
