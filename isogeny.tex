\documentclass{beamer}
%TODO:

%indentation?

% Choose how your presentation looks.
%
% For more themes, color themes and font themes, see:
% http://deic.uab.es/~iblanes/beamer_gallery/index_by_theme.html
%
\mode<presentation>
{
  \usetheme{default}      % or try Darmstadt, Madrid, Warsaw, AISEC ...
  \usecolortheme{default} % or try albatross, beaver, crane, ...
  \usefonttheme{default}  % or try serif, structurebold, ...
  \setbeamertemplate{navigation symbols}{}
  \setbeamertemplate{caption}[numbered]
} 

%\usepackage{fhgfont} funktioniert leider noch nicht
\usepackage[font=footnotesize]{caption} %for attributing pictures
\usepackage[english]{babel}
\usepackage[utf8x]{inputenc}
\usepackage{braket} % dirac notation
\usepackage{amsmath} % math symbols
\usepackage{amssymb} % other symbols
\graphicspath{ {img/} }
\usepackage{svg} %insert svg
\usepackage{svg-extract} %insert svg
\usepackage{graphicx} % insert pdfs
\newenvironment{rcases} % for right braces
{\left.\begin{aligned}}
	{\end{aligned}\right\rbrace}

\title[Shor's Algorithm]{SIKE - Supersingular Isogeny Key Encapsulation}
\author{Jonas von der Heyden}
\institute{FU Berlin}
\date{4.6.19}

\begin{document}
\newcommand{\source}[1]{\caption*{Source: {#1}} } %for attributing pictures
\begin{frame}
  \titlepage
\end{frame}

% Uncomment these lines for an automatically generated outline.
\begin{frame}{Outline}
  \tableofcontents
\end{frame}

\section{Introduction}

\begin{frame}{Introduction}

	\begin{itemize}
  		\item Goal of this presentation: Give a high-level overview of Shor's algorithm
	\end{itemize}

\end{frame}

\section{Mathematical primitive}

\subsection{Elliptic Curves}

\begin{frame}{Elliptic curves}
	%Weierstrass, Graphs
	content
\end{frame}

\begin{frame}{Elliptic Curve as Abelian Groups (G,+)}
	%...
	content
\end{frame}

\subsection{Isogenies}
\begin{frame}{Group homomorphisms}
	Example: $\phi: \mathbb{Z}\to\mathbb{Z}/3\mathbb{Z}$ is surjective and it's kernel consists of all elements in $\mathbb{Z}$ divisible by 3
\end{frame}

\subsection{m-torsion groups}
\subsection{Isogeny graphs}

\section{Cryptosystem}

\subsection{Diffie-Hellman}
\subsection{Diffie-Hellman as random walk}
\subsection{SIKE as random walk in Ramanujan graph}
\subsection{SIKE protocol}

\section{Areas of use}

\section{Security \& Complexity}
\subsection{SSDH assumption}
\subsection{Canetti Crawcyzk}

\section{Discussion}
\begin{frame}{Questions?}
...
\end{frame}

\end{document}
